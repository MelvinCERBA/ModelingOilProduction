\documentclass{article}
\usepackage[utf8]{inputenc}
\usepackage{graphicx,wrapfig,lipsum}
\usepackage{subfigure}
\usepackage{geometry}
\usepackage{amsmath}
% proportions du docc
 \geometry{
 a4paper,
 total={170mm,257mm},
 left=20mm,
 top=20mm,
 }
\title{Projet : modélisation de pics de production de ressources}
\author{Yoan Thomas\and  Aya Ismahene Kroussa\and Melvin Cerba}

\begin{document}
% Page de garde

\begin{figure}
        \center
        \includegraphics[scale = 0.4]{graphes/fig1.png}
\end{figure}
\maketitle
\begin{figure}[]
        \center
        \includegraphics[scale = 0.75]{graphes/Courbe_de_Hubbert_Data.png}
\end{figure}


% Sommaire
\newpage

\vspace*{\fill}
\tableofcontents
\vspace*{\fill}
%
\newpage

\section{Élément historique}
% Explication des motivations du rapport %
$\indent$ Peut-on prédire l'évolution de la production des ressources non renouvelables ? Pour tenter de répondre à cette question, nous proposons d'étudier à travers ce projet certains outils mathématiques qui permettent de modéliser la production des ressources non renouvelables. Plus spécifiquement, nous nous intéresserons à la production de pétrole.

\subsection{Modélisation des pics de productions}
% Contexte historique et pratique %
\begin{wrapfigure}{r}{5.5cm}
	\includegraphics[width=5.5cm]{graphes/Production.png}
	\includegraphics[width=5.5cm]{graphes/DataEtHubbert.png}
\end{wrapfigure} 
$\indent$ Être capable de prédire l'évolution de la production de diverses ressources a toujours été un enjeu majeur. Il est donc naturel que de nombreux modèles mathématiques promettant de modéliser cette évolution soient apparus. A travers ce projet, nous vous proposons d'étudier en détail l'un d'eux : la courbe de Hubbert. \\
$\indent$ En 1956, Marion King Hubbert présentait sa "courbe de Hubbert" à l'American Petroleum Institute. Son modèle, qui postule que la production  croit, atteint un unique pic, puis décroit au même rythme qu'elle a augmenté en premier lieu, ne fit pas beaucoup parler de lui à l'époque. Mais lorsqu'en 1971, conformément à ses prédictions, la production pétrolière américaine atteignit son maximum et commença à décliner, ses travaux furent réexaminés avec beaucoup plus d'intérêt. Les chocs pétroliers de 1973 et 1979 semblèrent cependant définitivement invalider son modèle, qui perdit rapidement l'attention de l'industrie pétrolière.  \\
$\indent$ Malgré tout, l'avènement du calcul informatique et la grande disponibilité de données poussèrent des auteurs modernes à exhumer le modèle de Hubbert et à l'étendre, notamment en donnant une formule mathématique à sa courbe, permettant ainsi de calculer son intégrale. C'est sur la base de ces nouveaux travaux que nous avons construit notre projet.\\ \\


\subsection{Courbe de Hubbert et sigmoïde}
% Introduction du modèle mathématique et explication des paramètres %
$\indent$ Les auteurs qui se sont rapproprié la courbe de Hubbert ont notamment travaillé à lui donner une expression mathématique. Ceci leur a permis de définir son intégrale, que nous appellerons "fonction sigmoïde" tout au long de ce rapport. Pour des raisons que nous expliciterons plus tard, cette dernière s'avère plus facile à manier que la courbe de Hubbert.\\
\\

\begin{figure}[h]
	\centering
    \subfigure{\includegraphics[width=0.40\textwidth]{graphes/CourbeHubbert.png}} 
    \subfigure{\includegraphics[width=0.40\textwidth]{graphes/Sigmoide.png}} 
    \caption{Une courbe de Hubbert et sa sigmoïde (son intégrale)}
    %\label{fig:foobar}
\end{figure}

$\indent$ Une \textbf{courbe de Hubbert} est une fonction à 4 paramètres : $a$,$b$ et $\tau$, qui définissent sa forme ; et $t$, qui représente le temps.

\begin{equation}\label{linspring}
H_{a,b,\tau}(t) = \frac{\frac{a b}{\tau} e^{-\frac{t}{\tau}}}{(1+b e^{-\frac{t}{\tau}})^2}
\qquad
\begin{gathered}
\includegraphics[width=0.6\textwidth]{graphes/Courbe_de_Hubbert_annotée.png}
\end{gathered}
\end{equation}
\\
$\indent$ Sa \textbf{sigmoïde} prend quant à elle les 4 paramètres suivants : $Q_{max}$,$t_*$ et $\tau$, qui définissent sa forme ; et $t$, qui représente le temps.
\\
\begin{equation}\label{linspring}
Q_{Q_{max},t_*,\tau}(t) = Q_{max} \frac{1}{1+ e^{-\frac{(t-t_*)}{\tau}}}
\qquad
\begin{gathered}
\includegraphics[width=0.6\textwidth]{graphes/SigmoideAnnotée.png}
\end{gathered}
\end{equation}

$\indent$ avec $\Delta$ la pente de la sigmoïde au niveau du point d'inflexion $t_*$, et :
\begin{align*}
 Q_{max} &= a \indent \indent \text{ le maximum de la sigmoïde}\\
 t_* &= \tau ln(b) \indent \text{le point d'inflexion de la sigmoïde}
\end{align*}


$\indent$ Introduction du modèle mathématique et explication des paramètres
\newpage
\section{Algorithme d'approximation}
% Explication de l'algorithme et des mathématiques sous jacentes %
\textit{Explication de l'algorithme et des mathématiques sous jacentes}

\subsection{Caractérisation}
% Explication détaillées de chaque parties de l'algorithme %
\textit{Explication détaillées de chaque parties de l'algorithme}

\subsubsection{Critère}
% Explication du critère à optimiser %
\textit{Explication du critère à optimiser}

\subsubsection{Gradient}
% Calcule du gradient et de la jacobienne %
\textit{Calcule du gradient et de la jacobienne}

\subsubsection{Direction de descente}
% Définition de la direction de descente %
\textit{Définition de la direction de descente}

\subsubsection{Figure des isocourbes avec les direction de descentes}
% Affichage de figure, et justification de la nécessité de l'utilisation d'une matrice de mise à l'échelle %
\textit{Affichage de figure, et justification de la nécessité de l'utilisation d'une matrice de mise à l'échelle}

\subsubsection{Matrice de mise à l'échelle}
% Définition de la matrice de mise à l'échelle %
\textit{Définition de la matrice de mise à l'échelle}

\subsubsection{Pseudo code}
% Pseudo Code %
\textit{Pseudo code}

\subsection{Test de contrôle de l'algorithme et des fonctions associées}
% Avant de lancer la l'algorithme sur des données réelles on vérifie que nos fonctions soient justes et que l'algorithme converge pour des données simulées %
\textit{Avant de lancer la l'algorithme sur des données réelles on vérifie que nos fonctions soient justes et que l'algorithme converge pour des données simulées}

\subsubsection{Vérification du gradient par différences finies}
% Test rapide du gradient par la méthode des différences finies %
\textit{Test rapide du gradient par la méthode des différences finies}

\subsubsection{Test de convergence avec données bruitées et non bruitées, en commençant plus ou moins loin de la solution}
% Test rapide de convergence l'algorithme sur des données générées, il y aura 4 tests comme précisé dans le titre se la sous-section %
\textit{Test rapide de convergence l'algorithme sur des données générées, il y aura 4 tests comme précisé dans le titre se la sous-section}

\subsection{Performance \textit{-optionnel}}
% Test des performances de l'algorithme %
\textit{Test des performances de l'algorithme}

\subsubsection{Temps de convergence selon diffèrent paramètre}
% Différents tests sur la vitesse de convergence sur par exemple :  le niveau de bruit, la distance de départ par rapport à la solution, facteur de rebroussement, etc %
\textit{Différents tests sur la vitesse de convergence sur par exemple : le niveau de bruit, la distance entre le point de départ et la solution, le facteur de rebroussement, etc }

\section{Données réels}
% Après l'introduction de l'algorithme, introduction de l'algorithme aux données réelles et évaluation des ces performances %
\textit{Après l'introduction de l'algorithme, introduction de l'algorithme aux données réelles et évaluation des ces performances}

\subsection{Vérification de la pertinence du modèle}
% Application de l'algorithme sur un jeu de données pertinentes %
\textit{Application de l'algorithme sur un jeu de données pertinentes}

\subsubsection{Vérification de la pertinence du modèle sur plusieurs set de donnés \textit{-optionnel}}
% Généralisation du test de pertinence %
\textit{Généralisation du test de pertinence}

\subsubsection{Vérification de la pertinence sur la somme de la production mondiale \textit{-optionnel}}
% Test de la pertinence du modèle sur une vue globale %
\textit{Test de la pertinence du modèle sur une vue globale}

\subsubsection{Explication des lacunes du modèle}
% Après avoir mit en exergue les lacunes sur quelques jeux de données, tentatives de caractérisations des hypothèses de validité du modèle %
\textit{Après avoir mit en exergue les lacunes sur quelques jeux de données, tentatives de caractérisations des hypothèses de validité du modèle}

\subsection{Pic de production \textit{-optionnel}}
% Vérification de la performance de l'algorithme pour trouver les pics de production %
\textit{Vérification de la performance de l'algorithme pour trouver les pics de production}

\subsubsection{Vérification de la performance de l'algorithme pour trouver les pics de production a posteriori}
% La détermination du pics étant simple a posteriori, on peut calculer de la performance de l'algorithme à trouver le pic de production a posteriori %
\textit{La détermination du pics étant simple a posteriori, on peut calculer de la performance de l'algorithme à trouver le pic de production a posteriori}

\subsubsection{Performance de l'algorithme pour trouver les pics de production sur les données qui peuvent être modélisées par une sigmoïde}
% En prenant des sets de données qui peuvent être convenablement modélisé par une sigmoïde, on peut cherche à quel point le modèle est prédictif. Pour cela on calcul la précision avec laquelle l'algorithme prédit le pic de production en se limitant aux premières données jusqu'à un temps $t$ %
\textit{En prenant des sets de données qui peuvent être convenablement modélisé par une sigmoïde, on peut cherche à quel point le modèle est prédictif. Pour cela on calcul la précision avec laquelle l'algorithme prédit le pic de production en se limitant aux premières données jusqu'à un temps $t$}

\subsection{Amélioration de modèle}
% Avec un algorithme flexible, on peut essayer d'améliorer la modélisation %
\textit{Avec un algorithme flexible, on peut essayer d'améliorer la modélisation}

\subsubsection{Deux sigmoïdes}
% Si il y a deux pics, est-ce que deux sigmoïdes additionnées est un bonne modélisation? %
\textit{Si il y a deux pics, est-ce que deux sigmoïdes additionnées est un bonne modélisation?}

\subsubsection{n sigmoïdes}
% Si il y a n pics, est-ce que n sigmoïdes additionnées est un bonne modélisation? %
\textit{Si il y a n pics, est-ce que n sigmoïdes additionnées est un bonne modélisation?}

\section{Conclusion}
\end{document}
